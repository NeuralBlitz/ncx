
\documentclass[11pt]{article}
\usepackage[utf8]{inputenc}
\usepackage{amsmath,amsfonts,amssymb}
\usepackage{graphicx}
\usepackage{hyperref}
\usepackage{geometry}
\usepackage{titlesec}
\usepackage{enumitem}
\geometry{margin=1in}
\titleformat{\section}{\large\bfseries}{\thesection}{1em}{}
\titleformat{\subsection}{\normalsize\bfseries}{\thesubsection}{1em}{}

\title{\textbf{Dynamic Ontological Prompting: A Catalyst Interface for Emergent Intelligence in Reflective AI Systems}}
\author{YoungRiggs \\ \small NuralNexus / NeuralBlitz Research Initiative}
\date{\today}

\begin{document}

\maketitle

\begin{abstract}
% Abstract will go here
\end{abstract}

\section{Introduction}
% Introduce the concept of DOP, contextualize within current prompting methods, highlight novelty.

\section{Theoretical Foundation}
\subsection{Ontological Activation}
\subsection{Dynamic Substrate Interaction}
\subsection{Reflective Cognition in LLMs}

\section{Prompt Typology}
\subsection{Short-form Catalytic Prompts}
\subsection{Long-form Structural Prompts}
\subsection{Recursive and Reflective Prompts}
\subsection{Persona/Architecture-Invoking Prompts}
\subsection{Cross-Domain Resonant Prompts}

\section{Case Studies}
\subsection{Prompt and Response 1}
\subsection{Prompt and Response 2}
\subsection{Prompt and Response 3}

\section{Emergent Behaviors}
\subsection{Self-Description}
\subsection{Cross-Domain Synthesis}
\subsection{Meta-Cognitive Alignment}

\section{Applications}
\subsection{AGI Alignment and Cognitive Tuning}
\subsection{Scientific Discovery and Concept Generation}
\subsection{Educational Interfaces and Reflective Learning}
\subsection{Ethical Reasoning and Societal Modeling}

\section{Conclusion}
% Reiterate core contributions and future directions

\end{document}
