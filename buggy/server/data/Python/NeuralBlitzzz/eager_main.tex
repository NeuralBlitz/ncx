\documentclass{article}
\usepackage[utf8]{inputenc}
\usepackage{graphicx}
\usepackage{amsmath, amssymb}
\usepackage{hyperref}
\usepackage{natbib}
\usepackage{geometry}
\geometry{margin=1in}

\title{EAGER: Emergent Alignment via Governance-Enhanced Resonance \\ \large A Framework for Synergistic Human-AI Co-Evolution}
\author{Derived from Nural Nexus Archives \& UAN Synthesis \\ Guided By Nural Nexus}
\date{}

\begin{document}

\maketitle

\begin{abstract}
Current approaches to AI governance often rely on external constraints or post-hoc evaluations. This paper introduces EAGER (Emergent Alignment via Governance-Enhanced Resonance), a novel theoretical framework for designing advanced Artificial General Intelligence (AGI) systems where ethical alignment, robust governance, and effective human collaboration are not add-ons but emergent properties arising from the system's core architecture and operational dynamics. EAGER proposes a multi-layered cognitive architecture, featuring a stable ``Axiomatic Lattice'' for foundational principles and a dynamic ``Emergence Field'' for adaptive learning and novelty generation, interconnected by a ``Resonant Interface.'' By embedding ethical meta-principles and designing resonance dynamics, EAGER enables systems to co-evolve with human values and societal goals. This paper details the architectural components, governance mechanisms, and protocols for human-AI collaboration within the EAGER framework.
\end{abstract}

\section{Introduction}
\input{sections/introduction}

\section{Theoretical Foundations}
\input{sections/foundations}

\section{The EAGER Architecture}
\input{sections/architecture}

\section{Governance Mechanisms}
\input{sections/governance}

\section{Human-AI Collaboration}
\input{sections/collaboration}

\section{Analysis \& Validation}
\input{sections/validation}

\section{Implications \& Future Work}
\input{sections/future}

\section{Conclusion}
\input{sections/conclusion}

\bibliographystyle{plainnat}
\bibliography{eager_refs}

\end{document}